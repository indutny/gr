%\documentclass[aps,prd,preprint]{revtex4-1}
\documentclass[aps,prd,final,twocolumn,floats,floatfix,nofootinbib,10pt]{revtex4-1}

% Note:  comment out one of the two documentclass commands, depending
% on whether you need the final or preprint version. It is most convenient
% to work in preprint mode, and switch to final only at the end.

\usepackage{graphicx}
\usepackage{amsmath}
\usepackage{amsfonts}
\usepackage{amssymb}
\usepackage{amstext}
\usepackage[english]{babel}
\usepackage{helvet}
\usepackage{microtype}
\usepackage{dsfont}
\usepackage[pdftex]{hyperref}
\usepackage{tikz}

\special{papersize=8.5in,11in}
\setlength{\pdfpageheight}{\paperheight}
\setlength{\pdfpagewidth}{\paperwidth}

\begin{document}

\title{Deriving Green's function for d'Alembert operator}
\author{indutny}
\date{\today}
\noaffiliation

\begin{abstract}
Spin-$\frac{1}{2}$ particles are governed by Dirac equation and are described by
4-component objects named spinors. Given the influence of Special Relativity
on the derivation of Dirac equation it is inevitable that the physical processes
involving spin-$\frac{1}{2}$ particles have to be independent of observer and their
inertial frame. Measurements made in one frame should agree with the
measurements made in another, and thus the components of spinor have to
transform between frames. In this paper we introduce the generators of
Lorentz transformations that form a Lie algebra, find the decomposition of
Lorentz algebra into a sum of two sub-algebras $\mathfrak{su}(2) \oplus \mathfrak{su}(2)$,
build left- and right- handed representations of spin-$\frac{1}{2}$ particles and combine
them to form a full Dirac spinor.
\end{abstract}

\maketitle

\section{Overview}

\subsection{Conventions}

The metric is:
\begin{equation}
\eta_{\mu\nu} = \eta^{\mu\nu} = \begin{pmatrix}
-1 & 0 & 0 & 0 \\
0 & 1 & 0 & 0 \\
0 & 0 & 1 & 0 \\
0 & 0 & 0 & 1
\end{pmatrix}.
\end{equation}

\subsection{Green's function}

Our goal is to solve:
\begin{equation}
\square \Psi(x - y) = \eta^{\mu\nu} \partial_\mu \partial_\nu \Psi(x - y) = \delta^{(4)} (x - y)
\end{equation}
for $\Psi(x - y)$.
The resulting function can be used to compute the solutions to more general differential equation:
\begin{equation}\label{eq:general-eq}
\square h(x) = T(x).
\end{equation}
Given the $\Psi(x - y)$ the reader can check that the solution to \eqref{eq:general-eq} is:
\begin{equation}
h(x) = \int d^4y \; \Psi(x - y) T(y).
\end{equation}

\section{Solution}

\end{document}