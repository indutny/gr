\documentclass[aps,prd,preprint]{revtex4-1}
%\documentclass[aps,prd,final,twocolumn,floats,floatfix,nofootinbib,10pt]{revtex4-1}

% Note:  comment out one of the two documentclass commands, depending
% on whether you need the final or preprint version. It is most convenient
% to work in preprint mode, and switch to final only at the end.

\usepackage{graphicx}
\usepackage{amsmath}
\usepackage{amsfonts}
\usepackage{amssymb}
\usepackage{amstext}
\usepackage[english]{babel}
\usepackage{helvet}
\usepackage{microtype}
\usepackage{dsfont}
\usepackage[pdftex]{hyperref}
\usepackage{tikz}

\special{papersize=8.5in,11in}
\setlength{\pdfpageheight}{\paperheight}
\setlength{\pdfpagewidth}{\paperwidth}

\begin{document}

\title{Solving problem in Carroll's Spacetime}
\author{indutny}
\date{\today}
\noaffiliation

\maketitle

\section{General Relativity Problem}

\subsection{Problem statement}

Prove that:
\begin{align}\label{eq:problem}
t_{\mu\nu} = \langle \left( \partial_\mu h_{\rho \sigma} \right) \left( \partial_\nu h^{\rho \sigma} \right) - \
  \frac{1}{2} \left( \partial_\mu h \right) \left( \partial_\nu h \right) - \
  \left( \partial_\rho h^{\rho \sigma} \right) \left( \partial_\mu h_{\nu \sigma} \right) - \
  \left( \partial_\rho h^{\rho \sigma} \right) \left( \partial_\nu h_{\mu \sigma} \right) \rangle
\end{align}
is invariant under gauge transformation:
\begin{align}\label{eq:metric-var}
\delta h_{\mu\nu} = \partial_\mu \xi_\nu + \partial_\nu \xi_\mu.
\end{align}
The ``$\langle$'', ``$\rangle$'' brackets mean time averaging and allow moving partial derivatives around:
\begin{align}
\langle A \partial_\mu B \rangle = - \langle \left( \partial_\mu A \right) B \rangle.
\end{align}

\subsection{First Attempt}

The variation of $h$ is:
\begin{align}\label{eq:trace-var}
\delta h =  \partial^{\mu\nu} \delta h_{\mu\nu} = 2 \partial^\lambda \xi_\lambda.
\end{align}

Let's simplify it term by term:
\begin{align}
& \left( \partial_\mu h_{\rho \sigma} \right) \left( \partial_\nu h^{\rho \sigma} \right)\\
= & \left( \partial_\mu \partial_\rho \xi_\sigma + \partial_\mu \partial_\sigma \xi_\rho \right)
    \left( \partial_\nu h^{\rho \sigma} \right) +
  \left( \partial_\mu h_{\rho \sigma} \right) \left( \partial_\nu \partial^\rho \xi^\sigma +
    \partial_\nu \partial^\sigma \xi^\rho \right) \\
= & -2 h^{\rho\sigma} \left( \partial_\mu \partial_\nu \partial_\rho \xi_\sigma +
    \partial_\mu \partial_\nu \partial_\sigma \xi_\rho \right), \label{eq:first-term}
\end{align}
\begin{align}
-\frac{1}{2} \left( \partial_\mu h \right) \left( \partial_\nu h \right) & =
  -\left( \partial_\mu \partial^\lambda \xi_\lambda \right) \left( \partial_\nu h \right) -
  \left( \partial_\mu h \right) \left( \partial_\nu \partial^\lambda \xi_\lambda \right) \\
& = 2 h^{\rho\sigma} \left( \eta_{\rho\sigma} \partial_\mu \partial_\nu \partial^\lambda \xi_\lambda \right),
  \label{eq:second-term}
\end{align}
\begin{align}
-\left( \partial_\rho h^{\rho \sigma} \right) \left( \partial_\mu h_{\nu \sigma} \right) & =
  -\left( \partial_\rho \partial^\rho \xi^\sigma + \partial_\rho \partial^\sigma \xi^\rho \right)
    \left( \partial_\mu h_{\nu \sigma} \right) -
  \left( \partial_\rho h^{\rho \sigma} \right) \left( \partial_\mu \partial_\nu \xi_\sigma +
    \partial_\mu \partial_\sigma \xi_\nu \right) \\
& = h^{\rho \sigma} \left[
  \eta_{\rho \nu} \left( \partial_\mu \square \xi_\sigma +
    \partial_\mu \partial_\sigma \partial^\lambda \xi_\lambda \right) +
    \partial_\rho \partial_\mu \partial_\nu \xi_\sigma +
    \partial_\rho \partial_\mu \partial_\sigma \xi_\nu \right], \label{eq:third-term}
\end{align}
and similarly:
\begin{align}
-\left( \partial_\rho h^{\rho \sigma} \right) \left( \partial_\nu h_{\mu \sigma} \right) =
h^{\rho \sigma} \left[
  \eta_{\rho \mu} \left( \partial_\nu \square \xi_\sigma +
    \partial_\nu \partial_\sigma \partial^\lambda \xi_\lambda \right) +
    \partial_\rho \partial_\mu \partial_\nu \xi_\sigma +
    \partial_\rho \partial_\nu \partial_\sigma \xi_\mu \right]. \label{eq:fourth-term}
\end{align}

Combining \eqref{eq:first-term}, \eqref{eq:second-term}, \eqref{eq:third-term}, and \eqref{eq:fourth-term} together
we gett:
\begin{align}
& h^{\rho\sigma} \Big[
  - 2 \left( \partial_\mu \partial_\nu \partial_\rho \xi_\sigma + \partial_\mu \partial_\nu \partial_\sigma \xi_\rho \right)
  + 2 \eta_{\rho\sigma} \partial_\mu \partial_\nu \partial^\lambda \xi_\lambda \notag \\
  & + \eta_{\rho \nu} \left( \partial_\mu \square \xi_\sigma +
    \partial_\mu \partial_\sigma \partial^\lambda \xi_\lambda \right) +
    \partial_\rho \partial_\mu \partial_\nu \xi_\sigma +
    \partial_\rho \partial_\mu \partial_\sigma \xi_\nu \notag \\
  & + \eta_{\rho \mu} \left( \partial_\nu \square \xi_\sigma +
    \partial_\nu \partial_\sigma \partial^\lambda \xi_\lambda \right) +
    \partial_\rho \partial_\mu \partial_\nu \xi_\sigma +
    \partial_\rho \partial_\nu \partial_\sigma \xi_\mu
\Big]
\end{align}

\end{document}
