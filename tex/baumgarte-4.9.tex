\documentclass[aps,prd,preprint]{revtex4-1}
%\documentclass[aps,prd,final,twocolumn,floats,floatfix,nofootinbib,10pt]{revtex4-1}

% Note:  comment out one of the two documentclass commands, depending
% on whether you need the final or preprint version. It is most convenient
% to work in preprint mode, and switch to final only at the end.

\usepackage{graphicx}
\usepackage{amsmath}
\usepackage{amsfonts}
\usepackage{amssymb}
\usepackage{amstext}
\usepackage{cancel}
\usepackage{mathtools}
\usepackage[english]{babel}
\usepackage{helvet}
\usepackage{microtype}
\usepackage{dsfont}
\usepackage[pdftex]{hyperref}
\usepackage{tikz}

\special{papersize=8.5in,11in}
\setlength{\pdfpageheight}{\paperheight}
\setlength{\pdfpagewidth}{\paperwidth}

\begin{document}

\title{4.9 in Baumgarte}
\author{indutny}
\date{\today}
\noaffiliation

\maketitle

\section{4.9}

\subsection{Useful Identities}

\begin{align} \label{eq:det-rel}
  \sqrt{-g} = \alpha \sqrt{\gamma},
\end{align}
\begin{align} \label{eq:det-derivative}
  \partial_t \ln \sqrt{\gamma} = -\alpha K + D_i \beta^i,
\end{align}
\begin{align} \label{eq:k-derivative}
  \partial_t K = -D^2 \alpha +
    \alpha \left( K_{ij} K^{ij} + 4 \pi (\rho + S) \right) +
    \beta^i D_i K,
\end{align}
\begin{align} \label{eq:g-upper}
  g^{ab} = \begin{pmatrix}
    -\alpha^{-2} & \alpha^{-2} \beta^i \\
    -\alpha^{-2} \beta^j & \gamma^{ij} - \alpha^{-2} \beta^i \beta^j
  \end{pmatrix}.
\end{align}

\subsection{Problem statement}

Using
\begin{align} \label{eq:christoffel-contraction}
  \prescript{(4)}{} \Gamma^a = -\frac{1}{\sqrt{-g}} \partial_b \left(
    \sqrt{-g} g^{ab}
  \right) = 0
\end{align}
derive following equalities:
\begin{align}
  (\partial_t - \beta^j \partial_j) \alpha & = -\alpha^2 K,
    \label{eq:first-result}\\
  (\partial_t - \beta^j \partial_j) \beta^i & =
    - \alpha^2 \left( \gamma^{ij} \partial_j \ln \alpha +
    \gamma^{jk} \Gamma^i_{jk} \right) \label{eq:second-result}.
\end{align}

\subsection{Solution}

It'd be useful to rewrite \eqref{eq:christoffel-contraction} for a
3-dimensional metric $\gamma$:
\begin{align} \label{eq:3d-christoffel}
  \Gamma^i = \gamma^{jk} \Gamma^i_{jk} =
    -\frac{1}{\sqrt{\gamma}} \partial_j \left(
      \sqrt{\gamma} \gamma^{ij} \right) =
    - \gamma^{ij} \partial_j \ln \sqrt{\gamma} - \partial_j \gamma^{ij}.
\end{align}

First let's rewrite the time component of
\eqref{eq:christoffel-contraction} using \eqref{eq:g-upper} and
\eqref{eq:det-rel}:
\begin{align}
  & \frac{1}{\alpha \sqrt{\gamma}} \partial_b (
    \alpha \sqrt{\gamma} g^{0b} ) =
  \frac{1}{\alpha \sqrt{\gamma}} \partial_t (
    -\sqrt{\gamma} \alpha^{-1} )
  + \frac{1}{\alpha \sqrt{\gamma}} \partial_i (
    \sqrt{\gamma} \alpha^{-1} \beta^i ) \\
  = &
    - \left(
      \alpha^{-2} \partial_t
      - \alpha^{-2} \beta^i \partial_i \right) \ln \sqrt{\gamma}
    + \left( \alpha^{-3} \partial_t
      - \alpha^{-3} \beta^i \partial_i \right) \alpha
    + \alpha^{-2} \partial_i \beta^i = 0 \\
  \implies &
    - \alpha \left(
      \partial_t
      - \beta^i \partial_i \right) \ln \sqrt{\gamma}
    + \left( \partial_t
      - \beta^i \partial_i \right) \alpha
    + \alpha \partial_i \beta^i = 0 \label{eq:time-pre-1} \\
  &
    \left( \partial_t - \beta^i \partial_i \right) \alpha =
    \alpha \left(
      \partial_t
      - \beta^i \partial_i \right) \ln \sqrt{\gamma}
    - \alpha \partial_i \beta^i \label{eq:time-1}.
\end{align}

Plugging \eqref{eq:det-derivative} into \eqref{eq:time-1}:
\begin{align}
  \left( \partial_t - \beta^i \partial_i \right) \alpha & =
    - \alpha^2 K + \alpha \left(
      D_i \beta^i
      - \beta^i \partial_i \ln \sqrt{\gamma}
      - \partial_i \beta^i \right) \\
  & =
    - \alpha^2 K + \alpha \gamma^{ij} \left(
        \partial_i \beta^i + \Gamma^i_{ij} \beta^j
        - \beta^i \partial_i \ln \sqrt{\gamma}
        - \partial_i \beta^i \right) \\
  & =
    - \alpha^2 K + \alpha \gamma^{ij} \left(
        \Gamma^i_{ij} \beta^j
        - \frac{1}{\sqrt{\gamma}}
          \beta^i \partial_i \sqrt{\gamma} \right) \\
  & =
    - \alpha^2 K + \alpha \gamma^{ij} \left(
        \Gamma^i_{ij} \beta^j
        - \frac{1}{\sqrt{\gamma}}
          \beta^i \frac{1}{2 \sqrt{\gamma}}
          \gamma \gamma^{jk} \partial_i \gamma_{jk} \right) \\
  & =
    - \alpha^2 K + \alpha \gamma^{ij} \beta^i \left(
        \Gamma^j_{ji}
      - \frac{1}{2} \gamma^{jk} \partial_i \gamma_{jk} \right) \\
  & =
    - \alpha^2 K + \alpha \gamma^{ij} \gamma^{jk} \beta^i \frac{1}{2} \left(
      \bcancel{\gamma_{ki,j}} + \cancel{\gamma_{kj,i}}
      - \bcancel{\gamma_{ij,k}} - \cancel{\gamma_{jk,i}} \right) \\
  & = -\alpha^2 K.
\end{align}
which is exactly \eqref{eq:first-result}.

For the spatial components of \eqref{eq:christoffel-contraction}:
\begin{align}
  \frac{1}{\alpha \sqrt{\gamma}} \partial_b (
    \alpha \sqrt{\gamma} g^{ib} ) & =
    \frac{1}{\alpha \sqrt{\gamma}} \partial_t (
      \alpha \sqrt{\gamma} g^{i0} )
    + \frac{1}{\alpha \sqrt{\gamma}} \partial_j (
      \alpha \sqrt{\gamma} g^{ij} ) \\
  & =
    \frac{1}{\alpha \sqrt{\gamma}} \partial_t (
      \alpha^{-1} \sqrt{\gamma} \beta^i )
    + \frac{1}{\alpha \sqrt{\gamma}} \partial_j \left(
      \alpha \sqrt{\gamma} \left(
        \gamma^{ij} - \alpha^{-2} \beta^i \beta^j \right) \right) \\
  & =
    \frac{1}{\alpha \sqrt{\gamma}} \partial_t (
      \alpha^{-1} \sqrt{\gamma} \beta^i )
    + \frac{1}{\alpha \sqrt{\gamma}} \partial_j \left(
    \alpha \sqrt{\gamma} \gamma^{ij} \right)
    - \frac{1}{\alpha \sqrt{\gamma}} \partial_j \left(
      \alpha^{-1} \sqrt{\gamma} \beta^i \beta^j \right).
      \label{eq:three-terms}
\end{align}

Let's work it term by term. First:
\begin{align}
  \frac{1}{\alpha \sqrt{\gamma}} \partial_t (
    \alpha^{-1} \sqrt{\gamma} \beta^i ) & =
    - \alpha^{-2} \beta^i \partial_t \ln \alpha
    + \alpha^{-2} \beta^i \partial_t \ln \sqrt{\gamma}
    + \alpha^{-2} \partial_t \beta^i \notag \\
  & =
    \alpha^{-2} \left(
      - \beta^i \partial_t \ln \alpha
      + \beta^i \partial_t \ln \sqrt{\gamma}
      + \partial_t \beta^i \right) \label{eq:first-term},
\end{align}
the second:
\begin{align}
  \frac{1}{\alpha \sqrt{\gamma}} \partial_j \left(
    \alpha \sqrt{\gamma} \gamma^{ij} \right) =
    \gamma^{ij} \partial_j \ln \alpha
    + \gamma^{ij} \partial_j \ln \sqrt{\gamma}
    + \partial_j \gamma^{ij} \label{eq:second-term},
\end{align}
and the third:
\begin{align}
  \frac{1}{\alpha \sqrt{\gamma}} \partial_j \left(
    \alpha^{-1} \sqrt{\gamma} \beta^i \beta^j \right) & =
    - \alpha^{-2} \beta^i \beta^j \partial_j \ln \alpha
    + \alpha^{-2} \beta^i \beta^j \partial_j \ln \sqrt{\gamma}
    + \alpha^{-2} \beta^j \partial_j \beta^i
    + \alpha^{-2} \beta^i \partial_j \beta^j \notag \\
  & =
    \alpha^{-2} \left(
      - \beta^i \beta^j \partial_j \ln \alpha
      + \beta^i \beta^j \partial_j \ln \sqrt{\gamma}
      + \beta^j \partial_j \beta^i
      + \beta^i \partial_j \beta^j \right) \label{eq:third-term}.
\end{align}

Combining \eqref{eq:first-term}, \eqref{eq:second-term},
\eqref{eq:third-term}:
\begin{align}
  \eqref{eq:three-terms} & = \alpha^{-2} \left(
    - \beta^i \left( \partial_t - \beta^j \partial_j \right) \ln \alpha
    + \beta^i \left(\partial_t - \beta^j \partial_j \right)
      \ln \sqrt{\gamma}
      + \left( \partial_t - \beta^j \partial_j \right) \beta^i
    - \beta^i \partial_j \beta^j \right) \notag \\
  & +
    \gamma^{ij} \partial_j \ln \alpha
    + \gamma^{ij} \partial_j \ln \sqrt{\gamma}
    + \partial_j \gamma^{ij} \label{eq:first-interim-1}.
\end{align}
We can use \eqref{eq:time-pre-1} to simplify further:
\begin{align}
  \alpha^2 \eqref{eq:first-interim-1} =
    \left( \partial_t - \beta^j \partial_j \right) \beta^i
    + \alpha^2 \left( \gamma^{ij} \partial_j \ln \alpha
      + \gamma^{ij} \partial_j \ln \sqrt{\gamma}
      + \partial_j \gamma^{ij} \right) = 0
\end{align}
and using \eqref{eq:3d-christoffel}:
\begin{align}
  \left( \partial_t - \beta^j \partial_j \right) \beta^i =
  - \alpha^2 \left(
    \gamma^{ij} \partial_j \ln \alpha
  - \Gamma^i \right)
\end{align}
which is \eqref{eq:second-result} up to a pesky sign.

\end{document}
