\documentclass[aps,prd,preprint]{revtex4-1}
%\documentclass[aps,prd,final,twocolumn,floats,floatfix,nofootinbib,10pt]{revtex4-1}

% Note:  comment out one of the two documentclass commands, depending
% on whether you need the final or preprint version. It is most convenient
% to work in preprint mode, and switch to final only at the end.

\usepackage{graphicx}
\usepackage{amsmath}
\usepackage{amsfonts}
\usepackage{amssymb}
\usepackage{amstext}
\usepackage[english]{babel}
\usepackage{helvet}
\usepackage{microtype}
\usepackage{dsfont}
\usepackage[pdftex]{hyperref}
\usepackage{tikz}

\special{papersize=8.5in,11in}
\setlength{\pdfpageheight}{\paperheight}
\setlength{\pdfpagewidth}{\paperwidth}

\begin{document}

\title{2.32 in Baumgarte}
\author{indutny}
\date{\today}
\noaffiliation

\maketitle

\section{Ricci tensor for conformal metric}

\subsection{Problem statement}

Compute Ricci tensor:
\begin{align}\label{eq:ricci}
R_{ij} = \partial_k \Gamma^k_{ij} - \partial_j \Gamma^k_{ik} + \Gamma^k_{ij} \Gamma^l_{kl} -
  \Gamma^k_{il} \Gamma^l_{jk}
\end{align}
for the metric
\begin{align}
\gamma_{ij} = \psi^4(r) \text{diag} \left(1, r^2, r^2 \sin^2 \theta \right),
\end{align}
where Christoffel Symbol is defined as:
\begin{align}\label{eq:cristoffel}
\Gamma^i_{jk} = \frac{1}{2} \gamma^{il} \left( \gamma_{lj,k} + \gamma_{lk,j} - \gamma_{jk,l} \right)
\end{align}
and $\psi$ is an arbitrary function of the radial coordinate $r$.

\subsection{Preparations}

The inverse of the diagonal metric is obviously:
\begin{align}
\gamma^{ij} = \frac{1}{\psi^4(r)} \text{diag} \left(1, \frac{1}{r^2}, \frac{1}{r^2 \sin^2 \theta} \right).
\end{align}

\subsection{Christoffel Symbols}

Let's start by meticulously computing all Christoffel symbols. With upper index ``$r$'' they are:
\begin{align}
%
% r_{rr}
%
\Gamma^r_{rr} & = \frac{1}{2} \gamma^{rr} \left(
  \gamma_{rr,r} + \gamma_{rr,r} - \gamma_{rr,r} \right) =
\frac{1}{2} \gamma^{rr} \gamma_{rr, r} =
\frac{1}{2 \psi^4} \partial_r \left( \psi^4 \right) =
\frac{2 \partial_r \psi}{\psi}, \\
%
% r_{r\theta}
%
\Gamma^r_{r \theta} & = \frac{1}{2} \gamma^{rr} \left(
  \gamma_{rr, \theta} + \gamma_{r \theta, r} - \gamma_{r \theta, r} \right) = 0, \\
%
% r_{r \phi}
%
\Gamma^r_{r \phi} & =
  \frac{1}{2} \gamma^{rr} \left( \gamma_{rr, \phi} + \gamma_{r\phi, r} - \gamma_{r \phi, r} \right) = 0, \\
%
% r_{\theta \theta}
%
\Gamma^r_{\theta \theta} & =
  \frac{1}{2} \gamma^{rr} \left(
    \gamma_{r \theta, \theta} + \gamma_{r \theta, \theta} - \gamma_{\theta \theta, r} \right) =
-\frac{1}{2 \psi^4} \partial_r \left( \psi^4 r^2 \right) =
-\frac{2 r^2 \partial_r \psi}{\psi} - r, \\
%
% r_{\theta \phi}
%
\Gamma^r_{\theta \phi} & =
  \frac{1}{2} \gamma^{rr} \left(
    \gamma_{r \theta, \phi} + \gamma_{r \phi, \theta} - \gamma_{\theta \phi, r} \right) = 0, \\
%
% r_{\phi \phi}
%
\Gamma^r_{\phi \phi} & =
  \frac{1}{2} \gamma^{rr} \left(
    \gamma_{r \phi, \phi} + \gamma_{r \phi, \phi} - \gamma_{\phi \phi, r} \right) =
-\frac{1}{2 \psi^4} \partial_r \left( \psi^4 r^2 \sin^2\theta \right) = \sin^2 \theta \; \Gamma^r_{\theta \theta}.
\end{align}

With upper index ``$\theta$'':
\begin{align}
%
% \theta_{rr}
%
\Gamma^\theta_{rr} & = \frac{1}{2} \gamma^{\theta \theta} \left(
  \gamma_{\theta r,r} + \gamma_{\theta r,r} - \gamma_{rr,\theta} \right) = 0, \\
%
% \theta_{r\theta}
%
\Gamma^\theta_{r \theta} & = \frac{1}{2} \gamma^{\theta\theta} \left(
  \gamma_{\theta r, \theta} + \gamma_{\theta \theta, r} - \gamma_{r \theta, \theta} \right) =
\frac{1}{2} \gamma^{\theta\theta} \gamma_{\theta \theta, r} =
\frac{1}{2 \psi^4 r^2} \partial_r \left( \psi^4 r^2 \right) =
-\frac{1}{r^2} \Gamma^r_{\theta \theta}, \\
%
% \theta_{r \phi}
%
\Gamma^\theta_{r \phi} & =
  \frac{1}{2} \gamma^{\theta \theta} \left(
    \gamma_{\theta r, \phi} + \gamma_{\theta \phi, r} - \gamma_{r \phi, \theta} \right) = 0, \\
%
% \theta_{\theta \theta}
%
\Gamma^\theta_{\theta \theta} & =
  \frac{1}{2} \gamma^{\theta \theta} \left(
    \gamma_{\theta \theta, \theta} + \gamma_{\theta \theta, \theta} - \gamma_{\theta \theta, \theta} \right) =
-\frac{1}{2 r^2 \psi^4} \partial_\theta \left( \psi^4 r^2 \right) = 0, \\
%
% \theta_{\theta \phi}
%
\Gamma^\theta_{\theta \phi} & =
  \frac{1}{2} \gamma^{\theta \theta} \left(
    \gamma_{\theta \theta, \phi} + \gamma_{\theta \phi, \theta} - \gamma_{\theta \phi, \theta} \right) = 0, \\
%
% \theta_{\phi \phi}
%
\Gamma^\theta_{\phi \phi} & =
  \frac{1}{2} \gamma^{\theta \theta} \left(
    \gamma_{\theta \phi, \phi} + \gamma_{\theta \phi, \phi} - \gamma_{\phi \phi, \theta} \right) =
-\frac{1}{2 r^2 \psi^4} \partial_\theta \left( \psi^4 r^2 \sin^2\theta \right) = - \sin \theta \cos \theta.
\end{align}


With upper index ``$\phi$'':
\begin{align}
%
% \phi{rr}
%
\Gamma^\phi_{rr} & = \frac{1}{2} \gamma^{\phi \phi} \left(
  \gamma_{\phi r,r} + \gamma_{\phi r,r} - \gamma_{rr,\phi} \right) = 0, \\
%
% \phi_{r\theta}
%
\Gamma^\phi_{r \theta} & = \frac{1}{2} \gamma^{\phi\phi} \left(
  \gamma_{\phi r, \theta} + \gamma_{\phi \theta, r} - \gamma_{r \theta, \phi} \right) = 0, \\
%
% \phi{r \phi}
%
\Gamma^\phi_{r \phi} & =
 \frac{1}{2} \gamma^{\phi \phi} \left(
    \gamma_{\phi r, \phi} + \gamma_{\phi \phi, r} - \gamma_{r \phi, \phi} \right) =
\frac{1}{2 \psi^4 r^2 \sin^2 \theta} \partial_r \left( \psi^4 r^2 \sin^2 \theta \right) = \Gamma^{\theta}_{r \theta},  \\
%
% \phi_{\theta \theta}
%
\Gamma^\phi_{\theta \theta} & =
  \frac{1}{2} \gamma^{\phi \phi} \left(
    \gamma_{\phi \theta, \theta} + \gamma_{\phi \theta, \theta} - \gamma_{\theta \theta, \phi} \right) = 0, \\
%
% \phi_{\theta \phi}
%
\Gamma^\phi_{\theta \phi} & =
 \frac{1}{2} \gamma^{\phi \phi} \left(
    \gamma_{\phi \theta, \phi} + \gamma_{\phi \phi, \theta} - \gamma_{\theta \phi, \phi} \right) =
\frac{1}{2 \psi^4 r^2 \sin^2 \theta} \partial_\theta \left( \psi^4 r^2 \sin^2 \theta \right) =
\frac{\cos \theta}{\sin \theta}, \\
%
% \phi_{\phi \phi}
%
\Gamma^\phi_{\phi \phi} & =
  \frac{1}{2} \gamma^{\phi \phi} \left(
    \gamma_{\phi \phi, \phi} + \gamma_{\phi \phi, \phi} - \gamma_{\phi \phi, \phi} \right) = 0.
\end{align}

Let's summarize the above computation by writing down non-zero components of $\Gamma^i_{jk}$:
\begin{align}
\Gamma^r_{rr} = \frac{2 \partial_r \psi}{\psi}, \;
\Gamma^r_{\theta \theta} = -\frac{2 r^2 \partial_r \psi}{\psi} - r, \;
\Gamma^r_{\phi \phi} = \sin^2 \theta \; \Gamma^r_{\theta \theta},
\\
\Gamma^\theta_{r \theta} = -\frac{1}{r^2} \Gamma^r_{\theta \theta}, \;
\Gamma^\theta_{\phi \phi} = - \sin \theta \cos \theta,
\\
\Gamma^\phi_{r \phi} = \Gamma^{\theta}_{r \theta}, \;
\Gamma^\phi_{\theta \phi} = \frac{\cos \theta}{\sin \theta}.
\end{align}

\subsection{Ricci Tensor}

\end{document}
