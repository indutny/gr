\documentclass[aps,prd,preprint]{revtex4-1}
%\documentclass[aps,prd,final,twocolumn,floats,floatfix,nofootinbib,10pt]{revtex4-1}

% Note:  comment out one of the two documentclass commands, depending
% on whether you need the final or preprint version. It is most convenient
% to work in preprint mode, and switch to final only at the end.

\usepackage{graphicx}
\usepackage{amsmath}
\usepackage{amsfonts}
\usepackage{amssymb}
\usepackage{amstext}
\usepackage[english]{babel}
\usepackage{helvet}
\usepackage{microtype}
\usepackage{dsfont}
\usepackage[pdftex]{hyperref}
\usepackage{tikz}
\usepackage[makeroom]{cancel}

\special{papersize=8.5in,11in}
\setlength{\pdfpageheight}{\paperheight}
\setlength{\pdfpagewidth}{\paperwidth}

\newcommand{\ogamma}{\overline{\gamma}}
\newcommand{\OGamma}{\overline{\Gamma}}
\newcommand{\OD}{\overline{D}}

\begin{document}

\title{3.3  in Baumgarte}
\author{indutny}
\date{\today}
\noaffiliation

\maketitle

\section{Ricci tensor for conformal metric}

\subsection{Problem statement}

Compute conformal Ricci tensor using non-conformal formula:
\begin{align}\label{eq:ricci}
R_{ij} = \partial_k \Gamma^k_{ij} - \partial_j \Gamma^k_{ik} + \Gamma^k_{ij} \Gamma^l_{kl} -
  \Gamma^k_{il} \Gamma^l_{jk}
\end{align}
for the metric
\begin{align}
\ogamma_{ij} = \psi^{-4} \gamma_{ij}, \; \ogamma^{ij} = \psi^{4} \gamma^{ij},
\end{align}
and Christoffel Symbol:
\begin{align}\label{eq:cristoffel}
\Gamma^i_{jk} = \frac{1}{2} \gamma^{il} \left( \gamma_{lj,k} + \gamma_{lk,j} - \gamma_{jk,l} \right),
\end{align}
where $\psi$ is an arbitrary function of spatial slice.

\subsection{Christoffel Symbol}

\begin{align}
\Gamma^{i}_{jk} & = \frac{1}{2} \ogamma^{il} \left(
  \ogamma_{lj,k} +
  \ogamma_{lk,j} -
  \ogamma_{jk,l} +
  4 \ogamma_{lj} \partial_k \ln \psi +
  4 \ogamma_{lk} \partial_j \ln \psi -
  4 \ogamma_{jk} \partial_l \ln \psi \right) \\
& = \OGamma^{i}_{jk} + 2 \left(
  \delta^{i}_{\; j} \partial_k \ln \psi + \delta^{i}_{\; k} \partial_j \ln \psi - \ogamma^{il} \ogamma_{jk} \partial_l \ln \psi
  \right) \\
& = \OGamma^{i}_{jk} + 2 \left(
  \delta^i_{\; j} \OD_k \ln \psi + \delta^i_{\; k} \OD_j \ln \psi - \ogamma_{jk} \ogamma^{il}  \OD_l \ln \psi \right).
\label{eq:christoffel-rel}
\end{align}
For future convenience (and to save electronic ink) we define $\alpha \equiv \ln \psi$, and thus
\eqref{eq:christoffel-rel} becomes:
\begin{align}
\Gamma^{i}_{jk} = \OGamma^{i}_{jk} + 2 \left(
  \delta^i_{\; j} \OD_k \alpha + \delta^i_{\; k} \OD_j \alpha - \ogamma_{jk} \ogamma^{il}  \OD_l \alpha \right).
\end{align}

\subsection{Ricci Tensor}

Let's introduce one more symbol to simplify later (formidable) calculations:
\begin{align}\label{eq:A}
A^i_{jk} & \equiv
  \delta^i_{\; j} \OD_k \alpha + \delta^i_{\; k} \OD_j \alpha - \ogamma_{jk} \ogamma^{il}  \OD_l \alpha, \\
\Gamma^i_{jk} & = \OGamma^i_{jk} + 2 A^i_{jk}.
\end{align}

With these definitions Ricci tensor becomes:
\begin{align}\label{eq:R-expanded}
R_{ij} = \; & \partial_k \left( \OGamma^k_{ij} + 2 A^k_{ij} \right) -
  \partial_j \left( \OGamma^k_{ik} + 2 A^k_{ik} \right) \notag \\
  & + \left( \OGamma^k_{ij} + 2 A^k_{ij} \right) \left( \OGamma^l_{kl} + 2 A^l_{kl} \right) -
  \left( \OGamma^k_{il} + 2 A^k_{il} \right) \left( \OGamma^l_{jk} + 2 A^l_{jk} \right) \\
= \; & \overline{R}_{ij} + 2 \partial_k A^k_{ij} -2 \partial_j A^k_{ik} \notag \\
  & + 2 \OGamma^k_{ij} A^l_{kl}  + 2 \OGamma^l_{kl} A^k_{ij} + 4 A^k_{ij} A^l_{kl} -
  2 \OGamma^k_{il} A^l_{jk} - 2 \OGamma^l_{jk} A^k_{il} - 4 A^k_{il} A^l_{jk}.
\end{align}

\subsection{\texorpdfstring{$O(A)$}{Single-A} terms}

Terms with two $A$ would involve two $\alpha$ and thus do not mix with terms with single $A$. We can therefore 
start by considering the terms with just a single $A$:
\begin{align}\label{eq:single-A}
2 \partial_k A^k_{ij} -2 \partial_j A^k_{ik} +
  2 \OGamma^k_{ij} A^l_{kl}  + 2 \OGamma^l_{kl} A^k_{ij} -
  2 \OGamma^k_{il} A^l_{jk} - 2 \OGamma^l_{jk} A^k_{il}.
\end{align}
The expression has a resemblance of a sum of covariant derivatives ($A^i_{jk}$ is trivially a tensor, symmetric under
interchange of lower indices $j$ and $k$):
\begin{align}\label{eq:cov-diff}
2 \OD_k A^k_{ij} - 2 \OD_j A^k_{ik} = & \;
  2 \partial_k A^k_{ij} + 2 \OGamma^k_{kl} A^l_{ij} - 2 \OGamma^l_{ki} A^k_{lj} -
    \cancel{2 \OGamma^l_{kj} A^k_{il}} \notag \\
  & - 2 \partial_j A^k_{ik} - 2 \OGamma^k_{jl} A^l_{ik} + 2 \OGamma^l_{ji} A^k_{lk} + \cancel{2 \OGamma^l_{jk} A^k_{il}} \\
= & \; 2 \partial_k A^k_{ij} - 2 \partial_j A^k_{ik} + 2 \OGamma^k_{kl} A^l_{ij} - 2 \OGamma^l_{ki} A^k_{lj} -
  2 \OGamma^l_{kj} A^k_{il} + 2 \OGamma^l_{ji} A^k_{lk},
\end{align}
and thus:
\begin{align}
\eqref{eq:single-A} = & \; 2 \OD_k A^k_{ij} - 2 \OD_j A^k_{ik} \\
= & \; 2 \OD_k \left(
    \delta^k_{\; i} \OD_j \alpha + \delta^k_{\; j} \OD_i \alpha - \ogamma_{ij} \ogamma^{kl} \OD_l \alpha
  \right) -
  2 \OD_j \left(
    \delta^k_{\; i} \OD_k \alpha + \delta^k_{\; k} \OD_i \alpha - \ogamma_{ik} \ogamma^{kl} \OD_l \alpha
  \right) \\
= & \; 2 \OD_i \OD_j \alpha + 2 \OD_j \OD_i \alpha - 2 \ogamma_{ij} \OD^2 \alpha - 2 \OD_j \OD_i \alpha -
  6 \OD_j \OD_i \alpha + 2 \OD_j \OD_i \alpha \\
= & \; 2 \OD_i \OD_j \alpha - 4 \OD_j \OD_i \alpha - 2 \ogamma_{ij} \OD^2 \alpha \label{prefinal-single-A}.
\end{align}

Expression above requires some further massaging:
\begin{align}
\OD_j \OD_i \alpha = \OD_j \partial_i \alpha = \partial_j \partial_i \alpha - \OGamma^l_{ji} \partial_l \alpha =
\OD_i \OD_j \alpha,
\end{align}
which we can substitute back into \eqref{prefinal-single-A} to get a nice expression:
\begin{align}\label{eq:final-single-A}
\eqref{eq:single-A} = -2 \left( \OD_i \OD_j \alpha + \ogamma_{ij} \OD^2 \alpha \right).
\end{align}

\subsection{\texorpdfstring{$O(A^2)$}{Double-A}  terms}

We still have two terms with double $A$ in them left:
\begin{align}\label{eq:double-A}
4 \left( A^k_{ij} A^l_{kl} - A^k_{il} A^l_{jk} \right)
\end{align}

The first term expands to
\begin{align}
A^k_{ij} A^l_{kl} = & \;
  \left( \delta^k_{\; i} \OD_j \alpha + \delta^k_{\; j} \OD_i \alpha - \ogamma_{ij} \ogamma^{km} \OD_m \alpha \right)
  \left( \delta^l_{\; k} \OD_l \alpha + \delta^l_{\; l} \OD_k \alpha - \ogamma_{kl} \ogamma^{lp} \OD_p \alpha \right) \\
= & \; \delta^k_{\; i} \delta^l_{\; k} \OD_j \alpha \OD_l \alpha + \delta^k_{\; j} \delta^l_{\; k} \OD_i \alpha  \OD_k \alpha -
  \delta^l_{\; k} \ogamma_{ij} \ogamma^{km} \OD_m \alpha \OD_l \alpha \notag \\
& + 3 \delta^k_{\; i} \OD_j \alpha \OD_k \alpha +3 \delta^k_{\; j} \OD_i \alpha \OD_k \alpha -
  3 \ogamma_{ij} \ogamma^{km} \OD_m \alpha \OD_k \alpha \notag \\
& - \delta^k_{\; i} \ogamma_{kl} \ogamma^{lp} \OD_j \alpha \OD_p \alpha -
  \delta^k_{\; j} \ogamma_{kl} \ogamma^{lp} \OD_i \alpha \OD_p \alpha +
  \ogamma_{ij} \ogamma^{km} \ogamma_{kl} \ogamma^{lp} \OD_m \alpha \OD_p \alpha \\
= & \; \OD_i \alpha \OD_j \alpha + \OD_i \alpha \OD_j \alpha - \ogamma_{ij} \ogamma^{kl} \OD_k \alpha \OD_l \alpha
  \notag \\
& + 3 \OD_i \alpha \OD_j \alpha + 3 \OD_i \alpha \OD_j \alpha - 3 \ogamma_{ij} \ogamma^{kl} \OD_k \alpha \OD_l \alpha
  \notag \\
& - \OD_i \alpha \OD_j \alpha - \OD_i \alpha \OD_j \alpha + \ogamma_{ij} \ogamma^{jk} \OD_k \alpha \OD_l \alpha \\
= & \; 6 \OD_i \alpha \OD_j \alpha - 3 \ogamma_{ij} \ogamma^{kl} \OD_k \alpha \OD_l \alpha. \label{eq:double-A-first}
\end{align}

The second term expands to
\begin{align}
A^k_{il} A^l_{jk}
\end{align}

\end{document}
